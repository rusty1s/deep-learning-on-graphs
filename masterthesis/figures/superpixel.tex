\begin{figure}[t]
\centering
\subfigure[\gls{SLIC}]{\includegraphics[scale=0.4]{bilder/slic_beispiel}}
\subfigure[Quickshift]{\includegraphics[scale=0.4]{bilder/quickshift_beispiel}}
\caption[\gls{SLIC} und Quickshift Beispielresultat]{Ein Bus segmentiert über (a) \gls{SLIC} mit jeweils 400, 800 und 1600 Superpixeln sowie über (b) Quickshift mit 600 Superpixeln~\cite{pascal_voc}.
Dabei werden die unterschiedlichen Verfahren zur Generierung von Superpixeln deutlich.
Wohingegen \gls{SLIC} möglichst quadratische, gleichgroße Superpixel erzeugt, generiert Quickshift sowohl sehr große wie auch sehr kleine Superpixel in allen möglichen Formvariationen.}
\label{fig:slic_quickshift}
\end{figure}

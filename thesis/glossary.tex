\newglossaryentry{R}{name=\ensuremath{\mathbb{R}}, description={Menge der reellen Zahlen}}
\newglossaryentry{R+}{name=\ensuremath{\mathbb{R}_+}, description={Menge der positiven reellen Zahlen inklusive Null}}
\newglossaryentry{N}{name=\ensuremath{\mathbb{N}}, description={Menge der natürlichen Zahlen}}
\newglossaryentry{diag}{name=\ensuremath{\text{diag}}, description={Diagonalfunktion}}
\newglossaryentry{I}{name=\ma{I}, description={Identitätsmatrix}}
\newglossaryentry{G}{name=\textcolor{red}{\ensuremath{G}}, description={Graph}}
\newglossaryentry{V}{name=\ensuremath{\mathcal{V}}, description={Knotenmenge ${\left\{v_i\right\}}^n_{i=1}$ eines Graphen \gls{G}}}
\newglossaryentry{E}{name=\ensuremath{\mathcal{E}}, description={Kantenmenge eines Graphen \gls{G} mit $\gls{E} \subseteq \gls{V} \times \gls{V}$}}
\newglossaryentry{adj}{name=\ensuremath{\sim}, description={Adjazenzrelation zweiter Knoten eines Graphen \gls{G} mit $u \gls{adj} v$ genau dann, wenn $u$ und $v$ adjazent}}
\newglossaryentry{w}{name=\ensuremath{w}, description={Gewichtsfunktion der Kanten eines Graph \gls{G} mit $\gls{w} \colon \gls{V} \times \gls{V} \to \gls{R+}$}}
\newglossaryentry{A}{name=\ma{A}, description={Adjazentmatrix eines Graphen \gls{G}}}
\newglossaryentry{degree}{name=\ensuremath{\deg}, description={Gradfunktion der Knoten eines Graphen \gls{G} mit $\gls{degree} \colon \gls{V} \to \gls{N}$}}
\newglossaryentry{d}{name=\ensuremath{d}, description={gewichtete Gradfunktion der Knoten eines Graphen \gls{G} mit $\gls{d} \colon \gls{V} \to \gls{R+}$}}
\newglossaryentry{D}{name=\ma{D}, description={gewichtete Gradmatrix}}
\newglossaryentry{p}{name=\ensuremath{p}, description={Positionsfunktion auf den Knoten \gls{V} mit $\gls{p} \colon \gls{V} \to \gls{R}^2$}}
\newglossaryentry{s}{name=\ensuremath{s}, description={kürzeste Distanzfunktion mit $\gls{s} \colon \gls{V} \times \gls{V} \to \gls{N}$}}
\newglossaryentry{L}{name=\ma{L}, description={Laplacian, unnormalisiert}}
\newglossaryentry{Lnorm}{name=\ma{\tilde L}, description={Laplacian, normalisiert}}
\newglossaryentry{Lboth}{name=\ma{\mathcal{L}}, description={Laplacian, normalisiert oder unnormalisiert}}
\newglossaryentry{lambda}{name=\ensuremath{\lambda}, description={Eigenwert}}
\newglossaryentry{Lambda}{name=\ma{\Lambda}, description={Diagonalmatrix der Eigenwerte des Laplacian}}
\newglossaryentry{ortho}{name=\ensuremath{\perp}, description={Orthogonalität}}
\newglossaryentry{eiv}{name=\ve{u}, description={Eigenvektor mit $\left\|\gls{eiv}\right\|_2 = 1$}}
\newglossaryentry{Eiv}{name=\ma{U}, description={Eigenvektormatrix}}

\section{Tschebyschow-Polynome}

\begin{itemize}
  \item bisheriger Ansatz skaliert nicht gut für große Graphen
  \item schneller Algorithmus zur Approximation des Filters notwendig $\Rightarrow$ Polynome niedriger Ordnung
  \item Größe des Filters soll unahängig zu den Daten sein
  \item approximiere $g(\mathcal{L})$ durch Polynom, dass rekursiv durch $\mathcal{L}$ berechnet werden kann
\end{itemize}

\emph{Tschebyschow-Polynome} (engl. \emph{Chebyshev}) bezeichnen eine Menge von Polynomen $T_n(x) \colon \mathbb{R} \to \mathbb{R}$ mit dem rekursiven Zusammenhang
\begin{equation}
  T_n(x) = 2x \cdot T_{n-1}(x) - T_{n-2}(x)
\end{equation}
mit $T_0(x) = 1$ und $T_1(x) = x$.
Ein Tschebyschow-Polynom $T_n$ ist ein Polynom $n$-ten Grads.
Diese Polynome formen eine Orthogonalbasis
\todo{stable recurrence property}

\subsection{Eigenschaften}

\begin{itemize}
  \item Für $x \in [-1, 1]$ gilt $T_k(x) \in [-1, 1]$
  \item Polynome formen eine Orthogonalbasis für $L^2 \left([-1, 1], \frac{d_x}{\sqrt{1-x^2}}\right)$, auch \emph{Hilbertraum} genannt
\end{itemize}

Rescale $\mathbf{\Lambda}$ zu $\mathbf{\tilde \Lambda} = \frac{2}{\lambda_{\max}} \mathbf{\Lambda} - \mathbf{I} \in {\left[-1, 1\right]}^{n \times n}$.
$\lambda_{\max}$ ist der Wert des größten Eigenvektors von $\mathbf{L}$.

Dann ist $T_k\left( \text{diag}^{-1}\left(\mathbf{\tilde \Lambda}\right) \right) \in {\left[-1, 1\right]}^{n}$

Hammon et al zeigen, dass $g_{\mathbf{\theta}}(\mathbf{\Lambda})$ beliebig genau mit Hilfe der  rekursiven Tschebyschow-Polynome mit Grad $k$ und einer festen Filtergröße $\mathbf{\theta}^{\prime} \in \mathbb{R}^k$ approximiert werden kann
\begin{equation}
  g_{\mathbf{\theta}}\left(\mathbf{\Lambda}\right) \approx g^{\prime}_{\mathbf{\theta}^{\prime}}\left(\mathbf{\Lambda}\right) = \text{diag}\left(\sum_{i = 0}^{k-1} \mathbf{\theta}^{\prime}_i \cdot T_i \left(\text{diag}^{-1}\left(\mathbf{\tilde \Lambda}\right)\right)\right)
\end{equation}

Beweisidee nachgucken

Es zeigt sich, dass
\begin{equation}
  y = \mathbf{U} \cdot g^{\prime}_{\mathbf{\theta}^{\prime}}\left(\mathbf{\Lambda}\right) \cdot \mathbf{U}^{\top} \cdot \mathbf{x} = g^{\prime}_{\mathbf{\theta}^{\prime}}\left(\mathbf{L}\right) \cdot \mathbf{x}
\end{equation}

wobei $g^{\prime}_{\mathbf{\theta}^{\prime}}\left(\mathbf{L}\right) = \sum_{i=0}^{k-1} \mathbf{\theta}_i T_i\left( \mathbf{\tilde L}\right)$ und $\mathbf{\tilde L} = \frac{2}{\lambda_{\max}}\mathbf{L} - \mathbf{I}$

Frage: Ist $\mathbf{\tilde L}$ auch im Intervall $-1$ und $1$????

Beweisidee: $\mathbf{L}^k = {\left(\mathbf{U}\mathbf{\Lambda}\mathbf{U}^{\top}\right)}^k = \mathbf{U}\mathbf{\Lambda}^k\mathbf{U^{\top}}$


Spektrale Filter, die repräsentiert werden durch ein Polynom vom Grad 

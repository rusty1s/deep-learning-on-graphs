$G = \left(\mathcal{V}, \mathcal{E}\right)$\\
$\mathcal{V} = {\left\{ v_i \right\}}^N_{i=1}$\\
$\left| V \right| = N < \infty$\\
Merkmalsfunktion $f_G \colon \mathcal{V} \to \mathbb{R}^m$\\
wenn nicht explizit aufgeführt, dann bla bla wir $f_G \colon \mathcal{V} \to \mathbb{R}$\\
Umschreibung in Tensor/Dense Matrix\\
$\mathcal{E} \subseteq \left\{ \left(u, v\right) \colon u, v \in \mathcal{V} \right\}$\\
Falls $\left( u, v \right) \in \mathcal{E}$, dann sind $u$ und $v$ adjazent und wir schreiben dann $u \sim v$\\
Gewichtsfunktion $w_G \colon \mathcal{V} \times \mathcal{V} \to \gls{R+}$\\
ungewichtet: $w_G \colon \mathcal{V} \times \mathcal{V} \to \left\{ 0, 1 \right\}$\\
Falls $\left( u, v \right) \notin \mathcal{E}$, dann $w_G\left(u, v\right) = 0$\\
Im ungewichteten Fall ist Gewichtsfunktion implizit durch $\mathcal{E}$ gegeben\\
Merkmalsfunktion und Gewichtsfunktion sind bei Betrachtung eines Graphen implizit gegeben \todo{umschreiben}

ungerichtet:
$u \sim v$ genau dann, wenn $v \sim u$ und
\begin{equation}
  w_G\left(u, v\right) = w_G\left(v, u\right)
\end{equation}
Fordern wir für den Verlauf dieser Arbeit (also keine gerichteten Graphen)

Schleife (für den weiteren Verlauf dieser Arbeit fordern wir $w_G\left(v, v\right) = 0$)\\

Der \emph{Grad} eines Knotens $v$ ist die Anzahl der Knoten, die adjazent zu ihm sind, d.h.
\begin{equation}
  \deg_G\left(v\right) = \sum_{v \sim u} 1
\end{equation}
Im Falle von gewichteten Graphen wird der Grad eines Knotens von $v$ auch oft über
\begin{equation}
  d_G\left(v\right) = \sum_{v \sim u} w_G\left(v, u\right)
\end{equation}
definiert.
Die unterschiedliche Notation macht deutlich, wann wir welchen Grad eines Knotens meinen.

Adjazenzmatrix, Umschreibung in Sparse Matrix/Tensor
Degreematrix, Umschreibung in Sparse Matrix/Tensor
Vorstellung $\text{diag}$

Ein Knoten $v \in \mathcal{V}$ eines Graphen $G$ heißt \emph{isoliert}, $d_G\left(v\right) = 0$.\\
Ein Graph heißt \emph{verbunden}, falls er keinen isolierten Knoten hat.\\

regulär, planar\\

Pfad, Pfadlänge
Digraph / Multiedge, Umschreibung in Tensor
connected / keine isolierten Knoten (Forderung für den Verlauf der Arbeit)

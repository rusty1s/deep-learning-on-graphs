\chapter{Grundlagen}

\section{Matrizen und Tensoren}

\section{Graphentheorie}

Graph Tupel $G = \left(\mathcal{V}, \mathcal{E}\right)$\\
$\mathcal{V} = {\left\{ v_i \right\}}^n_{i=1}$\\
$\left| V \right| = n < \infty$\\
Merkmalsfunktion $f_G \colon \mathcal{V} \to \mathbb{R}^m$\\
wenn nicht explizit aufgeführt, dann bla bla wir $f_G \colon \mathcal{V} \to \mathbb{R}$\\
Umschreibung in Tensor/Dense Matrix\\
$\mathcal{E} \subseteq \mathcal{V} \times \mathcal{V}$\\
Falls $\left( u, v \right) \in \mathcal{E}$, dann sind $u$ und $v$ adjazent und wir schreiben dann $u \sim v$\\
Gewichtsfunktion $w \colon \mathcal{V} \times \mathcal{V} \to \gls{R+}$\\
ungewichtet: $w \colon \mathcal{V} \times \mathcal{V} \to \left\{ 0, 1 \right\}$\\
Falls $\left( u, v \right) \notin \mathcal{E}$, dann $w\left(u, v\right) = 0$\\
Im ungewichteten Fall ist Gewichtsfunktion implizit durch $\mathcal{E}$ gegeben\\

ungerichtet:
$u \sim v$ genau dann, wenn $v \sim u$ und
\begin{equation}
  w\left(u, v\right) = w\left(v, u\right)
\end{equation}
Fordern wir für den Verlauf dieser Arbeit (also keine gerichteten Graphen)

Als \emph{Schleife} wird eine Kante bezeichnet, die einen Knoten mit sich selbst verbindet, d.h.\ $w\left(v, v\right) > 0$.
Ein Graph ohne Schleifen wird \emph{schleifenloser Graph} genannt.
Für den weiteren Verlauf dieser Arbeit fordern wir schleifenlose Graphen.\\

Adjazenzmatrix $\mathbf{A} \in \gls{R+}^{n \times n}$ eines Graphen $G$ mit $\mathbf{A}_{ij} = w(v_i, v_j)$\\
Wir sagen ein Knoten $v_i$ hat Position $i$ in $\mathbf{A}$.
Umschreibung in Sparse Matrix/Tensor\\

$G$ ist eindeutig definiert durch $\mathbf{A}$ und $f_G$.

Der \emph{Grad} eines Knotens $v$ ist die Anzahl der Knoten, die adjazent zu ihm sind, d.h.
\begin{equation}
  \deg\left(v\right) = \sum_{v \sim u} 1
\end{equation}
Im Falle von gewichteten Graphen wird der Grad eines Knotens von $v$ auch oft über
\begin{equation}
  d\left(v_i\right) = \sum_{j} \mathbf{A}_{ij}
\end{equation}
definiert.
Die unterschiedliche Notation macht deutlich, wann wir welchen Grad eines Knotens meinen.

Die Gradmatrix $\mathbf{D} \in \gls{R+}^{n \times n}$ eines Graphen $G$ ist definiert als Diagonalmatrix
\begin{equation}
  \mathbf{D} = \text{diag}\left( \left[ d\left(v_1\right), \ldots, d\left(v_n\right) \right] \right)
\end{equation}
Umschreibung in Sparse Matrix/Tensor

Ein Knoten $v \in \mathcal{V}$ eines Graphen $G$ heißt genau dann \emph{isoliert}, wenn $d\left(v\right) = 0$.\\
Ein Graph ist \emph{verbunden}, falls er keinen isolierten Knoten besitzt.
Für den weiteren Verlauf dieser Arbeit fordern wir, dass $G$ verbunden ist.\

Ein Graph heißt \emph{$k$-regulär} falls $\deg\left(v_i\right) = k$ für alle $1, \ldots, n$.
Ein \emph{ebener Graph} ist eine konkrete Darstellung eines Graphen auf der zweidimensionalen Ebene $\mathbb{R}^2$.
Jedem Knoten $v$ ist eine Positionsfunktion $p \colon \mathcal{V} \to \mathbb{R}^2$ zugeordnet, die die Position eines Knotens auf der Ebene eindeutig definiert.

Ein \emph{Weg} ist eine Folge von Knoten $\left( v_{x\left(1\right)}, v_{x\left(2\right)}, \ldots, v_{x\left(s\right)} \right)$, sodass $v_{x\left(i\right)} \sim v_{x\left(i+1\right)}$ für alle $1 \leq i < s$ mit Länge $s$, wobei $x \colon \left\{ 1, \ldots, n \right\} \to \left\{ 1, \ldots, n \right\}$ eine Permutation auf der Anzahl der Knoten.

Ein \emph{Pfad} ist ein Weg, sodass $v_i \neq v_{i+1}$.
Im Kontext von schleifenlosen Graphen sind die Begriffe Weg und Pfad äquivalent.
Wir schreiben $s\left(u, v\right)$ einer Funktion $s \colon \mathcal{V} \times \mathcal{V} \to \gls{N}$ für die Länge des kürzesten Pfades von $u$ nach $v$.

In Graphen mit \emph{Mehrfachkanten}, auch \emph{Multigraphen} genannt, können zwei Knoten durch mehrere Kanten verbunden sein.
Multigraphen lassen sich als Tensor über einen Vektor von Adjazenzmatrizen $\left[\mathbf{A}_1, \ldots, \mathbf{A}_m\right] \in \gls{R+}^{m \times n \times n}$ schreiben.

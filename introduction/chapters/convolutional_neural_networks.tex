\subsection{Convolutional Neural Networks}

\begin{itemize}
  \item nutzt die räumliche Anordnung von Bildern aus
  \item besteht aus drei grundlegenden Ideen:
  \begin{enumerate}
    \item \textbf{Local Receptive Fields:}
    \begin{itemize}
      \item anstatt ein Bild (z.B. $28 \times 28$) als einen langen vertikalen Vektor ($784$ Komponenten) zu beschreiben, wird es als eine Matrix aus Neuronen betrachtet
      \item nicht jedes Eingabe-Neuron wird mit jedem Hidden Neuron in der ersten Schicht verbunden, sondern nur eine kleine Region der Eingabe (z.B. $5 \times 5$)
      \item diese Region heißt \emph{local receptive field} für das Hidden-Neuron
      \item die Region wird pixelweise von links nach rechts und von oben nach unten bewegt und jeweils mit einem Hidden-Neuron verbunden ($28 \times 28$ Bild, $5 \times 5$ Region führt zu $24 \times 24$ Hidden-Neuronen)
      \item es kann auch nicht pixelweise verschoben werden (z.B. um 2 Pixel)
    \end{itemize}
    \item \textbf{Shared Weights and Biases:}
    \begin{itemize}
      \item jedes Neuron besitzt einen Bias und $5 \times 5$ Gewichte zu seinem Local Reeptive Field
      \item jedes Neuron in der ersten Schicht besitzt die gleichen Gewichte und den gleichen Bias!
      \item $\Rightarrow$ alle Neuronen in der ersten Schicht sollen das gleiche Merkmal lernen, nur an anderen Positionen
      \item Die Verbindung zwischen Eingabe und erstem Hidden-Layer heißt daher auch oft \emph{Feature Map}
      \item ein Convolutional Layer kann aus mehreren Feature Maps bestehen
    \end{itemize}
    \item \textbf{Pooling Layers:}
    \begin{itemize}
      
    \end{itemize}
  \end{enumerate}
\end{itemize}

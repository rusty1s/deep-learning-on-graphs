\RequirePackage{ifthen}

\newcommand \Arbeitsbezeichnung{Master-Thesis}
\newcommand \Autor{Matthias Fey}
\newcommand \Arbeitstitel{Convolutional Neural Networks auf Graphrepr{\"a}sentationen von Bildern}
\newcommand \Erstgutachter{Prof.~Dr.~Heinrich~M{\"u}ller}
\newcommand \Zweitgutachter{M.Sc.~Jan~Eric~Lenssen}
\newcommand \Lehrstuhl{Lehrstuhl Informatik VII}
\newcommand \Lehrstuhltitel{Graphische Systeme}
\newcommand \Uni{TU Dortmund}

\RequirePackage{ifpdf} \ifpdf
  \pdfoutput=1
  \pdftrue
  \message{pdfLaTeX}
  \documentclass[pdftex,12pt,a4paper,twoside,ngerman,numbers=noenddot]{scrbook}
  \usepackage{float}
  \usepackage[pdftex]{thumbpdf}
  \usepackage[pdftex]{graphicx}
  \usepackage[pdftex]{hyperref}
  \usepackage{pdfpages}
  \pdfoutput=1
  \pdfcompresslevel=9
  \DeclareGraphicsExtensions{.pdf,.jpg,.png}
\else
  \pdffalse
  \message{LaTeX}
  \documentclass[dvips,12pt,a4paper,twoside,ngerman,numbers=noenddot]{scrbook}
  \usepackage{float}
  \usepackage{graphicx}
  \usepackage{epsf}
  \usepackage[dvips]{hyperref}
  \DeclareGraphicsExtensions{.eps}
\fi


% Informationen fuer pdf-File festlegen
\hypersetup
{
    pdfauthor = {\Autor},
    pdftitle = {\Arbeitstitel},
    pdfsubject = {\Arbeitsbezeichnung, TU Dortmund, Fakult{\"a}t f{\"u}r Informatik},
    pdfproducer = {LaTeX},
    pdfview = FitV,
    pdfstartview = FitV,
    pdfhighlight = /I,
    pdfborder = 0 0 0,
    colorlinks = false,
    bookmarksopen,
    bookmarksopenlevel = 1,
    bookmarksnumbered = false,
    plainpages = false
}%


% Seitenformat anpassen
\usepackage[a4paper,left=3.5cm,right=2.5cm,bottom=3.5cm,top=3cm]{geometry}
\setlength{\headheight}{15pt}
% -------------------------------------------------------------------
% Grafikpakete einbinden
\usepackage{amsmath,amssymb}
\usepackage{flafter}
\usepackage{subfigure}
\usepackage{pdfpages}

% -------------------------------------------------------------------
\usepackage{ifthen}

% -------------------------------------------------------------------
\usepackage[absolute,overlay]{textpos}
\setlength{\TPHorizModule}{1mm}
\setlength{\TPVertModule}{\TPHorizModule}
\textblockorigin{0mm}{0mm}
\usepackage{fix-cm}
\usepackage{setspace}
\usepackage{scrhack}
% -------------------------------------------------------------------
% Korrekte Darstellung der Umlaute
\usepackage[german,ngerman]{babel}
\usepackage[utf8]{inputenc}
\usepackage[T1]{fontenc}
\usepackage{ae,aecompl}


% -------------------------------------------------------------------
% Bibtex deutsch
\usepackage[numbers,sort]{natbib}


% -------------------------------------------------------------------
% Anführungszeichen
\usepackage[babel,german=quotes]{csquotes}


% -------------------------------------------------------------------
% URLs
\usepackage{url}
% Trennung langer urls
\usepackage[hyphenbreaks]{breakurl}
\def\UrlBreaks{\do\a\do\b\do\c\do\d\do\e\do\f\do\g\do\h\do\i\do\j\do\k\do\l%
\do\m\do\n\do\o\do\p\do\q\do\r\do\s\do\t\do\u\do\v\do\w\do\x\do\y\do\z\do\0%
\do\1\do\2\do\3\do\4\do\5\do\6\do\7\do\8\do\9\do\-}%

% -------------------------------------------------------------------
% Caption anpassen
\usepackage[margin=0pt,font=small,labelfont=bf]{caption}

% -------------------------------------------------------------------
% Erweitere Tabellen
\usepackage{booktabs}

% -------------------------------------------------------------------
% Zeilenabstand einstellen
\renewcommand{\baselinestretch}{1.25}
% Floating-Umgebungen anpassen
\renewcommand{\topfraction}{0.9}
\renewcommand{\bottomfraction}{0.8}

% -------------------------------------------------------------------
% Keine einzelnen Zeilen beim Anfang eines Abschnitts (Schusterjungen)
\clubpenalty=10000
% Keine einzelnen Zeilen am Ende eines Abschnitts (Hurenkinder)
\widowpenalty=10000
\displaywidowpenalty=10000

\parindent=0cm

% -------------------------------------------------------------------
% Kopfzeile hinzufuegen
\usepackage{fancyhdr}
\usepackage{extramarks}

\pagestyle{fancy}
\renewcommand{\chaptermark}[1]{\markboth{#1}{}}
\renewcommand{\sectionmark}[1]{\markright{#1}{}}

\fancyhf{}
\fancyhead[LE,RO]{\thepage}
\fancyhead[RE]{\textit{\nouppercase{\leftmark}}}
\fancyhead[LO]{\textit{\nouppercase{\rightmark}}}

\fancypagestyle{plain}{ %
\fancyhf{} % remove everything
\renewcommand{\headrulewidth}{0pt} % remove lines as well
\renewcommand{\footrulewidth}{0pt}} \pagestyle{headings}



% -------------------------------------------------------------------
% Eigene Farben definieren
\usepackage{color}
\definecolor{TUGreen}{rgb}{0.517,0.721,0.094}
\definecolor{TUOrange}{rgb}{1.0,0.7176,0.0}
\definecolor{BrightGray}{gray}{0.9}
\definecolor{DarkGray}{gray}{0.2}
\definecolor{white}{rgb}{1,1,1}
\definecolor{black}{rgb}{0,0,0}
\definecolor{red}{rgb}{1,0,0}




% -------------------------------------------------------------------
% Programm-Listings einbinden und formatieren
\usepackage{listings}

\lstdefinestyle{C++}
{
language=C++,
backgroundcolor=\color{BrightGray},
keywordstyle=\texttt\bfseries,  %\color{TUGreen}\bfseries,
commentstyle=\color{DarkGray},
stringstyle=\color{red},
showstringspaces=false,
basicstyle=\small\color{black},
numbers=left,
captionpos=b,
tabsize=4,
breaklines=true
}


% -------------------------------------------------------------------
% Algorithmen
\usepackage[plain,chapter]{algorithm}
\usepackage{algorithmic}

\usepackage{enumerate}

% -------------------------------------------------------------------
% Algorithmen anpassen
\renewcommand{\algorithmicrequire}{\textit{Eingabe:}}
\renewcommand{\algorithmicensure}{\textit{Ausgabe:}}
\floatname{algorithm}{Algorithmus}
\renewcommand{\listalgorithmname}{Algorithmenverzeichnis}
\renewcommand{\algorithmiccomment}[1]{\color{grau}{// #1}}

\section{Spektraler Faltungsoperator}
\label{spektraler_faltungsoperator}

Wir interessieren uns dafür ein Signal auf den Knoten eines Graphes $f \colon \gls{V} \to \gls{R}$ zu verarbeiten.
Dabei kann die Signalfunktion auf den Knoten auch als ein Vektor $\ve{f} \in \gls{R}^N$ verstanden werden mit $\ve{f}_i = f\left(\gls{v}_i\right)$.

\todo{bild mit signalen}
Signal $f \colon \gls{V} \to \gls{R}$ auf Graphen\\
Signal kann ebenso als Vektor $\ve{f} \in \gls{R}^n$ aufgefasst werden (impliziert Ordnung der Knoten, ist durch Adjazenzmatrix gegeben)\todo{schon weiter oben definiert, eher in grundlagen unterbringen?}\\

\cite{Shuman}
\cite{Defferrard}
\cite{Hammond}

Transformation eines Signals in dessen Frequenzspektrum $f \colon \gls{R} \to \gls{R}$

Die komplexe Exponentialfunktion $\exp\left(\mathrm{i}\omega t\right)$ der Fourier-Transformation beschreibt die \emph{Eigenfunktionen} des eindimensionalen Laplace-Operators $\nabla^2$.
\todo{das verstehe ich nicht}

Die inverse Fourier-Transformation
\begin{equation}
  f\left(t\right) = \frac{1}{2\pi} \int \hat f\left(\omega\right) \exp\left(\mathrm{i}\omega t\right)\,\mathrm{d}\omega
\end{equation}
kann damit als die Ausdehnung von $f$ in Bezug auf die Eigenfunktionen des Laplace-Operators verstanden werden.

\subsection{Graph-Fourier-Transformation}
\label{graph_fourier_transformation}

\paragraph{Die klassische Fourier-Transformation}
\label{fourier_transformation}
Die klassische Fourier-Transformation
\\\\
Analog lässt sich die \emph{Graph-Fourier-Transformation} einer Funktion $\ve{f} \in \gls{R}^N$ auf den Knoten eines Graphen \gls{G} als Ausdehnung von \ve{f} in Bezug auf die Eigenvektoren der Laplace-Matrix definieren~\cite{Shuman}:
\begin{equation}
  \hat f\left(\gls{lambda}_i\right) \coloneqq \left\langle \ve{f}, \gls{eiv}_i \right\rangle \text{ \bzw{} } \ve{\hat f} \coloneqq \gls{Eiv}^{\top}\ve{f}.
  \label{eq:graph_fourier_transformation}
\end{equation}
Die inverse Graph-Fourier-Transformation ergibt sich dann als~\cite{Shuman}
\begin{equation}
  f\left(\gls{v}_i\right) = \sum_{j=1}^N \hat f\left(\gls{lambda}_j\right) {\left(\gls{eiv}_j\right)}_i\text{ \bzw{} }\ve{f} = \gls{Eiv}\ve{\hat f}.
  \label{eq:inverse_graph_fourier_transformation}
\end{equation}

Die Graph-Fourier-Transformation~\eqref{eq:graph_fourier_transformation} und ihre Inverse~\eqref{eq:inverse_graph_fourier_transformation} bieten uns eine Möglichkeit ein Signal in zwei unterschiedlichen Domänen zu repräsentieren, nämlich die Knotendomäne (das unveränderte Signal auf der Knotenmenge $\ve{f}$) und der spektralen Domäne (das transformierte Signal in das Spektrum des Graphen $\ve{\hat f}$).
Dabei berücksichtigt eine Transformation eines Signals in der spektralen Domäne des Graphen die Struktur eines Graphen, wohingegen eine solche Transformation eines Signals auf den Knoten nicht möglich wäre.

\todo{parseval relation?}

Spectral Graph Domain

\begin{itemize}
  \item \emph{Spectral Graph Domain}: Der Raum der Eigenfunktionen von $\mathcal{L}$
  \item Analogon (Nachbildung) einer \emph{Fourier-Transformation} von Funktionen auf gewichteten Graphen
\end{itemize}

Das ermöglicht uns das Anwenden verschiedener Operation wie Filterung, Translation oder Faltung.\todo{quelle}

Directly extending this construction to arbitrary weighted graphs is problematic, as it is unclear
how to define scaling and translation on an irregular graph.
We approach this problem by working in the spectral graph domain, i.e.\ the space of eigenfunctions of the graph Laplacian \gls{Lboth}.
This tool from spectral graph theory, provides an analogue of the Fourier transform for functions on weighted graphs.

Eigenwerte werden als Frequenz aufgefasst, die das Spektrum des Graphen beschreiben.
Die Eigenvektoren beschreiben beschreiben die Signale zu den gegebenen Frequenzen.\todo{mmh}

Fourier-Transformation beschreibt die gleiche Funktion $f$, aber in einer völlig anderen Domäne.
Nicht in der Vertex-Domäne, sondern in der Spectrum-Domäne, d.h.\ auf Basis der Eigenwerte.

In classical Fourier analysis, the eigenvalues carry a specific notion of frequency:
for $\omega$ close
to zero (low frequencies), the associated complex exponential
eigenfunctions are smooth, slowly oscillating functions,
whereas for $\omega$ far from zero (high frequencies), the associated
complex exponential eigenfunctions oscillate much more
rapidly. In the graph setting, the graph Laplacian eigenvalues
and eigenvectors provide a similar notion of frequency. For
connected graphs, the Laplacian eigenvector u0 associated
with the eigenvalue 0 is constant and equal to 1
at each
vertex. The graph Laplacian eigenvectors associated with low
frequencies $\gls{lambda}_2$ vary slowly across the graph; i.e., if two
vertices are connected by an edge with a large weight, the
values of the eigenvector at those locations are likely to be
similar. The eigenvectors associated with larger eigenvalues
oscillate more rapidly and are more likely to have dissimilar
values on vertices connected by an edge with high weight.

Graph Fourier Transformation und ihre Inverse gibt uns die Möglichkeit, ein Signal in zwei verschiedenen Domänen zu repräsentieren, nämlich die Knotendomäne (das unveränderte Signal auf der Knotenmenge $\ve{f}$) und der spektralen Domäne (das transformierte Signal in das Spektrum des Graphen).
Signale, die im Spektrum definiert werden, werden \emph{Kernel} genannt.\todo{ok?}

Die Fourier-Transformation wird unter anderem gerne genutzt, da sie gute Eigenschaften hat.

\subsection{Polynomielle Approximation}
\label{polynomielle_approximation}

\paragraph{Tschebyschow-Polynome}
\label{tschebyschow_polynome}

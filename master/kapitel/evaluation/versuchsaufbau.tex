\section{Versuchsaufbau}
\label{versuchsaufbau}

Generelle Netzstruktur, softmax auf Klassen abgebildet
Conv mit MaxPool gefolgt von AveragePool auf Fully Connect auf Softmax

\subsection{Datensätze}
\label{datensaetze}

Die vorgestellten Faltungsmethoden aus Kapitel~\ref{raeumliches_lernen} und~\ref{spektrales_lernen} \bzgl{} des Lernens auf zweidimensionalen euklidischen Graphen wurden über einer Reihe von Datensätzen verifiziert, die im Folgenden vorgestellt werden.
Dafür wurden die Bildermengen in eine Superpixelrepräsentation (\gls{SLIC} und Quickshift) konvertiert und darauf basierend in eine Graphrepräsentation transformiert (\vgl{} Kapitel~\ref{graphrepraesentationen_von_bildern}).
Zusätzlich zu der Präsentation der Datensätze enthält dieses Kapitel damit insbesondere die Parameterwahl der jeweiligen Superpixelalgorithmen, welche jeweils händisch über einer Untermenge der Bilder eines jeden Datensatzes ermittelt wurden, und weiterhin die entsprechenden Merkmalsselektionen, die nach dem beschriebenen Prinzip aus Kapitel~\ref{merkmalsselektion} errechnet wurden.

\paragraph{MNIST}
\label{mnist}

Der \emph{\gls{MNIST}} Datensatz enthält eine große Menge eindeutig klassifizierter handgeschriebener Zahlen von $0$ bis $9$, welcher daher zum Lernen einer Schrift- \bzw{} Zahlenerkennung genutzt werden kann~\cite{mnist}.
Er besteht aus $55000$ Trainingsbildern, $5000$ Validierungsbildern sowie $10000$ Testbildern.
Die Bilder des Datensatzes sind einheitlich auf die Größe $28 \times 28$ skaliert und besitzen lediglich einen Farbkanal mit Grauwerten, welcher angibt, ob ein Pixel des Bildes zu einer Zahl (weiß), zu deren Rand oder zum Hintergrund (schwarz) gehört~\cite{mnist}.
Aufgrund seiner kleinen Datengröße und leichten Handhabung gilt er als die ideale Einführung in Prinzipien des maschinellen Lernens und zeichnet sich damit als ideal für die Verifizierung eines neuen Ansatzes \bzgl{} neuronaler Netze aus.
Insbesondere kann der Datensatz während des gesamten Trainings im Hauptspeicher gehalten werden, was den Aufwand \bzgl{} der Verarbeitung und Eingabe der Daten auf ein Minimum reduziert.

Die ermittelten Parameter \bzgl{} der beiden benutzten Superpixelalgorithmen sind in Tabelle~\ref{tab:mnist} gegeben.
Für \gls{SLIC} sind das die Parameter $K \in \gls{N}$, \dhe{} die Anzahl der gewünschten Segmente, sowie $F \in \gls{R}$ für die Gewichtung zwischen der Form und den Farbabgrenzungen der Superpixel.
Für Quickshift ergeben sich dagegen drei wählbare Parameter — $\gls{sigma} \in \gls{R}$ für die Wahl der Standardabweichung der Gaußfunktion, $\alpha \in \gls{R}$ für die Gewichtung des Farbterms sowie $S \in \gls{N}$ zur Einschränkung der Berechnung über ein Fenster der Größe $S \times S$.
Für eine detaillierte Beschreibung der Parameter sei auf Kapitel~\ref{superpixel_verfahren} verwiesen.
\begin{table}[t]
\centering
\begin{tabular}{rlrlrlrlrlrl}
  \toprule
  \multicolumn{6}{c}{\gls{SLIC}} & \multicolumn{6}{c}{Quickshift}\\
  \midrule
  $K$ & $100$ & $F$ & $5$ & & & $\gls{sigma}$ & $2$ & $\alpha$ & $1$ & $S$ & $2$\\
  \midrule
  $\overline{N}$ & $64.6$ & $N_{\min}$ & $50$ & $N_{\max}$ & $80$ & $\overline{N}$ & $82.1$ & $N_{\min}$ & $5$ & $N_{\max}$ & $154$\\
  $\overline{\gls{degree}}$ & $5.7$ & $\gls{degree}_{\min}$ & $1$ & $\gls{degree}_{\max}$ & $19$ & $\overline{\gls{degree}}$ & $6.8$ & $\gls{degree}_{\min}$ & $1$ & $\gls{degree}_{\max}$ & $101$\\
  \bottomrule
\end{tabular}
\caption[\gls{MNIST} Superpixelparameter]{Wahl der Superpixelparameter des \gls{MNIST} Datensatzes.}
\label{tab:mnist}
\end{table}

Aus der Wahl der Superpixelparameter ergeben sich die ebenfalls in der Tabelle datierten Werte der durchschnittlichen, minimalen und maximalen Anzahl an Knoten $\overline{N}$, $N_{\min}$ \bzw{} $N_{\max}$ sowie dem durchschnittlichen, minimalen und maximalen Knotengrad $\overline{\gls{degree}}$, $\gls{degree}_{\min}$ \bzw{} $\gls{degree}_{\max}$ über der Menge aller aus den Bildern generierten Graphen bei einer Konnektivität von $8$.
Wohingegen \gls{SLIC} über alle Bilder relativ gleich große Knotenmengen mit relativ gleichem Knotengrad erzeugt, kann dies bei Quickshift je nach Bild stark variieren.
So erzeugt Quickshift in dem \gls{MNIST} Datensatz \bspw{} große schwarze Bereiche für den Hintergrund, die dementsprechend auch einen sehr hohen Knotengrad besitzen.
Bei \gls{SLIC} werden stattdessen auch die gleichfarbigen, schwarzen Flächen in einheitliche Intervalle unterteilt.
Abbildung~\ref{fig:mnist} veranschaulicht die beiden Superpixel- \bzw{} Graphrepräsentationen anhand eines Bildes aus dem \gls{MNIST} Datensatz.
\section{MNIST}

Trainingsbilder: 55.000

\begin{itemize}
  \item 10.000 Steps mit Batch Size 64 (ungefähr 12 Epochen)
  \item Learning Rate 0.001
  \item klassisches Convolution Neural Network nachgebildet mit Gridgraphen
  \item Conv1: $5 \times 5$, $1 \rightarrow 32$
  \item MaxPool1: Size 2, Stride 2
  \item Conv2: $5 \times 5$, $32 \rightarrow 64$
  \item MaxPool2: Size 2, Stride 2
  \item FC1: 1024
  \item Dropout: 0.5
  \item FC2: 10
\end{itemize}

\subsection{Auswertung}

\begin{itemize}
  \item \textbf{2D Conv > Max:} 0.18s pro Batch, Accuracy: 99.189, Cost: 0.03458
  \item \textbf{2D Conv > 2D Conv > Max:} 0.25s pro Batch, Accuracy: 99.139, Cost: 0.03062
  \item \textbf{Chebyshev $k=25$ GCNN:} 0.91s pro Batch, Accuracy: 98.888, Cost: 0.04329
  \item \textbf{$k=1$ GCNN:} 0.22s pro Batch, Accuracy: 96.765, Cost: 0.10596
  \item \textbf{Partitioned GCNN:}
  \begin{itemize}
    \item Conv > Max: 0.45s pro Batch, Accuracy: 98.998, Cost: 0.03198
    \item Conv > Conv > Max: 2.87s pro Batch, Accuracy: 99.189, Cost: 0.02704
  \end{itemize}
\end{itemize}

\subsection{SLIC}

\begin{itemize}
  \item keine lokale Normierung
  \item Stddev: $1$
  \item 4 Level
  \item Graphkonnektivität: $1$
  \item Anzahl Segmente: $100$
  \item Compactness: $10$
  \item Maximum Iterations: $10$
  \item Sigma: $0$
  \item Anzahl Partitionen: 8
  \item Features: Area, Bbox height, bbox width, Mean Color = $4$ Features
  \item \textbf{Aufbau}: Conv zu 32, Pool2, Conv zu 64, Pool2, Conv zu 128, Pool2, Conv zu 256, Pool2, AveragePool, FC210
  \item Meiste zeit wird durch Partitionierung verschwendet.
  \item \textbf{Ergebnisse}: 0.79s pro Batch, Accuracy: 0,79497, Loss: 0.62814
  \item enttäuschend!
\end{itemize}

\section{PascalVOC}

erster Test:
17 s Preprocess, 12s Training auf BatchSize 64
loss = 0.2, acc = 0.55


Die in Kapitel~\ref{merkmalsselektion} beschriebene Merkmalsselektion reduziert für \gls{MNIST} die Menge an Formmerkmalen (\vgl{} Kapitel~\ref{merkmalsextraktion}) im ersten statistisch basierten Test auf $12$ Merkmale, welche im zweiten Schritt rekursiv auf $9$ Merkmale reduziert werden.
Die ermittelten (unterschiedlichen) Formmerkmale für \gls{SLIC} und Quickshift \bzgl{} \gls{MNIST} sind in Tabelle~\ref{tab:mnist_merkmale} aufgezeigt.
\begin{table}[t]
\centering
\begin{tabular}{lccccccccc}
  \toprule
  \gls{SLIC} & $\hat{x}$ & $\hat{y}$ & $\mathrm{ecc}$ & $\mathrm{dia}$ & $\mathrm{ext}$ & $\gls{mu}^{\prime}_{20}$ & $\gls{lambda}_2$ & $\mathrm{axis}_1$ & $\mathrm{axis}_2$\\
  Quickshift & $\hat{x}$ & $\mathrm{ecc}$ & $\mathrm{dia}$ & $\mathrm{ext}$ & $\gls{mu}_{03}$ & $\gls{mu}_{21}$ & $\gls{mu}_{30}$ & $\gls{eta}_{03}$ & $\mathrm{ori}$\\
  \bottomrule
\end{tabular}
  \caption[\gls{MNIST} Merkmalsselektion]{Merkmalsselektion des \gls{MNIST} Datensatzes auf $9$ Formmerkmale.}
\label{tab:mnist_merkmale}
\end{table}
Wohingegen sich die Merkmalsselektion bei \gls{SLIC} eher für Formmerkmale entscheidet, die aus den Momenten gewonnen werden können, so genießen bei Quickshift die reinen translationsinvarianten Momente \gls{mu} größeres Interesse.
Daraus ergeben sich $10$ Merkmale eines Knotens in \gls{MNIST} inklusive der Durchschnittsfarbe eines Superpixels.

\paragraph{CIFAR-10}
\label{cifar_10}

Der \emph{\gls{Cifar}} Datensatz, auch \gls{Cifar}-10 genannt, besteht aus $60000$ farbigen Bildern, die jeweils genau einer von $10$ Klassen zugeordnet sind~\cite{cifar_10}.
$45000$ Bilder werden dabei als Trainingsbilder, $5000$ als Validierungsbilder und $10000$ als Testbilder genutzt.
Zu jeder Klasse existieren genau $6000$ Bilder, welche gleichmäßig auf die Bilduntermengen aufgeteilt sind.
Die Klassen der Bilder sind im Folgenden: Flugzeug, Auto, Vogel, Katze, Reh, Hund, Frosch, Pferd, Schiff und Lastwagen.
Die Bilder der Klassen sind dabei \emph{einander ausschließend}.
So enthält die Klasse \enquote{Auto} nur kleinere Personenwagen, wohingegen die Klasse \enquote{Lastwagen} auch nur als solche zu klassifizierenden Fahrzeuge enthält~\cite{cifar_10}.
Die Bilder haben eine einheitliche Größe von $32 \times 32$ Pixeln und besitzen drei Farbkanäle.
Sie passen aufgrund ihrer Größe ähnlich zu \gls{MNIST} komplett in den Hauptspeicher.
Aufgrund dessen ist er für eine Bildklassifizierung sehr beliebt, da er schnelle Trainingszeiten garantiert und dabei trotzdem alle Techniken des Deep Learnings ausgeschöpft werden müssen, um qualitativ hochwertige Resultate zu erzielen.
Der \gls{Cifar} Datensatz liegt ebenfalls in einer zweiten Version vor, genannt \gls{Cifar}-100, der 60000 Bilder in 100 Klassen unterteilt, welcher aber in dieser Arbeit keine Verwendung findet~\cite{cifar_10}.

\begin{table}[t]
\centering
\resizebox{\textwidth}{!}{%
\begin{tabular}{rlrlrlrlrlrl}
  \toprule
  \multicolumn{6}{c}{\gls{SLIC}} & \multicolumn{6}{c}{Quickshift}\\
  \midrule
  $K$ & $200$ & $F$ & $5$ & & & $\gls{sigma}$ & $1$ & $\alpha$ & $1$ & $S$ & $5$\\
  \midrule
  $\overline{N}$ & $232.1$ & $N_{\min}$ & $186$ & $N_{\max}$ & $263$ & $\overline{N}$ & $182.0$ & $N_{\min}$ & $18$ & $N_{\max}$ & $624$\\
  $\overline{\gls{degree}}$ & $6.3$ & $\gls{degree}_{\min}$ & $1$ & $\gls{degree}_{\max}$ & $21$ & $\overline{\gls{degree}}$ & $7.4$ & $\gls{degree}_{\min}$ & $1$ & $\gls{degree}_{\max}$ & $67$\\
  \bottomrule
\end{tabular}}
\caption[\gls{Cifar}-10 Superpixelparameter]{Wahl der Superpixelparameter des \gls{Cifar}-10 Datensatzes.}
\label{tab:cifar_10}
\end{table}

\begin{figure}[t]
\centering
\subfigure[\gls{SLIC}]{%
  \includegraphics[width=\textwidth]{bilder/cifar_10_slic.png}
}
\subfigure[Quickshift]{%
  \includegraphics[width=\textwidth]{bilder/cifar_10_quickshift.png}
}
  \caption[\gls{Cifar}-10]{Bild des \gls{Cifar}-10 Datensatzes~\cite{cifar_10}, jeweils dargestellt als (1) Originalbild, (2) Superpixelrepräsentation, (3) Durchschnittsfarbe der Superpixel und (4) als generierter Graph.}
\label{fig:cifar_10}
\end{figure}


\todo{datenreduktion}

\paragraph{PASCAL VOC}
\label{pascal_voc}

\emph{\gls{Pascal}} Datensatz
\cite{pascal_voc}

\begin{table}[t]
\centering
\resizebox{\textwidth}{!}{%
\begin{tabular}{rlrlrlrlrlrl}
  \toprule
  \multicolumn{6}{c}{\gls{SLIC}} & \multicolumn{6}{c}{Quickshift}\\
  \midrule
  $K$ & $1600$ & $F$ & $30$ & & & $\gls{sigma}$ & $2$ & $\alpha$ & $0.75$ & $S$ & $8$\\
  \midrule
  $\overline{N}$ & $1540.9$ & $N_{\min}$ & $1082$ & $N_{\max}$ & $1839$ & $\overline{N}$ & $2131.6$ & $N_{\min}$ & $234$ & $N_{\max}$ & $29010$\\
  $\overline{\gls{degree}}$ & $6.5$ & $\gls{degree}_{\min}$ & $1$ & $\gls{degree}_{\max}$ & $50$ & $\overline{\gls{degree}}$ & $7.7$ & $\gls{degree}_{\min}$ & $1$ & $\gls{degree}_{\max}$ & $256$\\
  \bottomrule
\end{tabular}}
\caption[\gls{Pascal} Superpixelparameter]{Wahl der Superpixelparameter des \gls{Pascal} Datensatzes.}
\label{tab:pascal_voc}
\end{table}

\begin{figure}[t]
\centering
\subfigure[\gls{SLIC}]{%
  \includegraphics[width=\textwidth]{bilder/pascal_voc_slic.png}
}
\subfigure[Quickshift]{%
  \includegraphics[width=\textwidth]{bilder/pascal_voc_quickshift.png}
}
\caption[\gls{Pascal}]{Bild aus dem \gls{Pascal} Datensatz, jeweils dargestellt als (1) Superpixelrepräsentation, (2) Durchschnittsfarbe der Superpixel und (3) Graphrepräsentation.}
\label{fig:pascal_voc}
\end{figure}


datenreduktion

\subsection{Metriken}
\label{metriken}

Loss Function
Accuracy

\subsection{Parameterwahl}
\label{parameterwahl}

Vorstellung aller Parameter
Was gibt es denn hier überhaupt?
Dropout, L2 Regularisierung?
BatchSize?
Globale/normale Lokalisierung
Standardabweichung für Gauß
LEARNING RATE, LEARNING RATE DECAY

Alle Faltungen wurden dabei mit einer Partitionsgröße von $8$ bei $K=0$ und $K=1$ implementiert, um ein \gls{CNN} mit einem $3 \times 3$ Filter zu simulieren.
Es erscheint jedoch vorstellbar die Filtergröße bei größerer lokaler Kontrollierbarkeit, \dhe{} $K > 1$, weiter zu reduzieren und die Gefahr des Overfittings damit aufgrund der kleineren Anzahl an Trainingsparametern einzuschränken.

\paragraph{Datenreduktion}
\label{datenreduktion}

\subsection{Augmentierung von Graphen}
\label{augmentierung_von_graphen}

hier auf die Formeln von TensorFlow referenzieren, d.h. TensorFlow Quelle angeben
\cite{tensorflow}

Augmentierung auf Graphen über left/right
Farbanpassungen


nesser ist es, dass Bild vorher zu ändern, da sich dadurch die Superpixelrepräsentation ändert
und folglich zu realisiterischer Augmentierung führt.

\subsection{Vorverarbeitung und Eingabe der Daten}
\label{vorverarbeitung}

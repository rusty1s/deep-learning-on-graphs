\chapter{Einleitung}
\label{einleitung}

Homepage\footnote{\url{https://github.com/rusty1s/embedded\_gcnn}}

Insbesondere hier schon auf räumlich und spektral eingehen, da es woanders keinen sinn macht

In the context of the generalization of Convolutional Neural Networks (CNNs) to irregular domains modelled by graphs, the problem is the classification or regression of graph signals, i.e.\ signals who take a value at each vertex of a weighted graph. There is two approaches to convolve a graph signal with a (learned) filter:
Spatially sliding a filter on the vertices, as you would slide a filter on a 2D image or a 1D audio signal (the pixels form a grid graph, the time forms a line graph), i.e.\ the straightforward application of the definition of a convolution. This approach however presents two challenges: (1) the definition of a receptive field / neighbourhood, because sampling on arbitrary graphs is not necessarily uniform and (2), the ordering of nodes, because problem-specific ordering, e.g.\ spatially ordered pixels or time ordered samples, is missing. These recent works, who present the same ideas differently, spatially define the convolution operator on graphs:
The spectral formulation builds on Spectral Graph Theory and Computational Harmonic Analysis. It decomposes the graph Laplacian (via an eigendecomposition) to form a Fourier basis, which properties are analog to the classical Fourier basis on n-dimensional Euclidean spaces. A convolution in the graph domain is then equivalent to a multiplication in the spectral domain. See [1211.0053] for an overview of the Graph Signal Processing field. Note that the spectral approach was also proposed in [1312.5851] (using FFTs) to speed up CNNs on regular 2D Euclidean spaces (i.e.\ for images). The limitations of this approach is (for now, we have plans to address those): (1) filters are rotation invariant, and (2) filters are not directly transferrable to a different graph. These recent works, who build on one another, spectrally define the convolution operator on graphs:

\begin{figure}[t]
\centering
\includegraphics[width=\textwidth]{bilder/problemstellung.png}
\caption[Problemstellung]{Illustration der verfolgten Problemstellung in dieser Arbeit.
Bilder werden für ihre Eingabe in ein neuronales Netz zuvor in eine korrespondierende Graphrepräsentationen konvertiert, auf denen gelernt wird.}
\label{fig:problemstellung}
\end{figure}

\section{Aufbau der Arbeit}
\label{aufbau_der_arbeit}

Die vorliegende Arbeit gliedert sich, neben den in Kapitel~\ref{grundlagen} vorgestellten Grundlagen, die für das weitere Verständnis der Arbeit von Nöten sind, in vier Bereiche:

Das Kapitel~\ref{graphrepraesentationen_von_bildern} widmet sich der Gewinnung einer Graphrepräsentation aus einem Bild.
Dabei werden insbesondere zwei Verfahren zur Generierung eines Graphen aus einem Bild vorgestellt — die herkömmliche Gitterrepräsentation eines Bildes (Kapitel~\ref{gitter}) sowie die Gewinnung eines Graphen aus einer vorberechneten Superpixelrepräsentation (Kapitel~\ref{superpixel}).

Kapitel~\ref{raeumliches_lernen} und~\ref{spektrales_lernen} erläutern die beiden unterschiedlichen Ansätze des graphbasierten Lernens in neuronalen Netzen — das räumliche und das spektrale Lernen — getrennt voneinander.
Beide Verfahren umfassen den aktuellen Stand der wissenschaftlichen Forschung auf diesem Gebiet.
Zu jedem Ansatz finden sich in den Unterkapiteln~\ref{raeumliche_erweiterung} \bzw{}~\ref{gcn_erweiterung} entwickelte Erweiterungen \bzgl{} der beschriebenen Problemstellung.

Kapitel~\ref{evaluation} evaluiert die vorgestellten sowie entwickelten Ansätze anhand der gegebenen Problemstellung im Vergleich zueinander sowie im Vergleich zu klassischen Lösungen auf diesem Gebiet.
Dazu wird zunächst in Unterkapitel~\ref{versuchsaufbau} der Versuchsaufbau erläutert und dessen Ergebnisse in den Unterkapiteln~\ref{ergebnisse} und~\ref{laufzeitanalyse} präsentiert.
In Kapitel~\ref{ausblick} folgt ein abschließender Ausblick zu möglichen weiterführenden Arbeiten.


\section{Mathematische Notationen}
\label{mathematische_notationen}

% Tensor, was bedeutet z.B. $\gls{W}_i$

% Diagonalmatrix
% skalarprodukt

Ein Bild kann folglich durch einen dreidimensionalen Tensor $\gls{B} \in {\left[0, 1\right]}^{H \times W \times C}$ repräsentiert werden, wobei $H, W \in \gls{N}$ die Höhe \bzw{} Breite des Bildes angeben und $C \in \left\{1, 3\right\}$ die Anzahl der Farbkanäle des Bildes beschreibt, \dhe{} ein Graubild mit nur einem Kanal oder ein Farbbild mit drei Kanälen (\zB{} über das RGB-Farbmodell).
Die Werte der einzelnen Farbkanäle des Bildes werden dabei auf $\left[0, 1\right]$ skaliert.

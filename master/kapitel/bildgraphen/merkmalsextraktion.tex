\subsection{Merkmalsextraktion}
\label{merkmalsextraktion}

Die Darstellung eines Bildes über einen Graphen \gls{G}, der aus einer Superpixelrepräsentation \gls{Smenge} gewonnen wurde, besitzt in der Regel weitaus weniger Knoten im Gegensatz zu der reinen Darstellung des Bildes über eine Gitterrepräsentation.
Die Superpixel \bzw{} die Regionen der Segmentierungsmaske können dabei jedoch die willkürlichsten Formen annehmen und besitzen lediglich die Einschränkung, dass diese stets zusammenhängend sind.
Die Form eines Superpixels muss demnach bestmöglichst eingefangen \bzw{} beschrieben werden können — ein Prozess, der in der Bildverarbeitung als \emph{Merkmalsextraktion} bekannt ist~\cite{momente}.
Ein geeigntes Mittel zur Beschreibung einzelner Objekte in einem segmentierten Bild sind die \emph{Momente}, welche in nicht-zentrierte, translationsinvariante, skalierungsinvariante und rotationsinvariante Momente unterschieden werden~\cite{momente}.

\paragraph{Nicht-zentrierte Momente}
\label{nicht_zentrierte_momente}

Zu der binären Segmentierungsmaske $\gls{Smaske} \in {\left\{0, 1\right\}}^{H \times W}$ sind die \emph{nicht-zentrierten Momente} vom Grad $\left(i+j\right)$, $i,j\in\gls{N}$, definiert als~\cite{momente}
\begin{equation*}
  \gls{M}_{ij} \coloneqq \sum_x^W \sum_y^H x^i y^j \gls{Smaske}_{yx}.
\end{equation*}
Obwohl der Grad eines Moments beliebig hoch gewählt werden kann, so reichen in der Praxis meist wenige Momente niedrigen Grades aus ($\le 3$), um eine Region hinreichend genau zu charakterisieren~\cite{momente}.
Bildeigenschaften, die durch nicht-zentrierte Momente beschrieben werden können, sind unter anderem dessen Fläche über $\gls{M}_{00}$ sowie dessen absoluter Schwerpunkt $\left\{\bar{x}, \bar{y}\right\} = \left\{ \gls{M}_{10}/\gls{M}_{00}, \gls{M}_{01}/\gls{M}_{00}\right\}$~\cite{momente}.

\paragraph{Translationsinvariante (zentrale) Momente}
\label{translationsinvariante_momente}

Nicht-zentrierte Momente sind aufgrund ihrer Berücksichtigung der Position einer Region im Bild meist unerwünscht, sie helfen aber für die weitere Definition von translationsinvarianten Momenten.
Mit Hilfe der absoluten Schwerpunktskoordinaten $\left\{ \bar{x}, \bar{y} \right\}$ können die \emph{translationsinvarianten Momente} über
\begin{equation*}
  \gls{mu}_{ij} \coloneqq \sum_x^W \sum_y^H {\left(x - \bar{x}\right)}^i {\left(y - \bar{y}\right)}^j \gls{Smaske}_{yx}.
\end{equation*}
definiert werden~\cite{momente}.
Sie lassen sich weiterhin direkt aus $\gls{M}_{ij}$ ermitteln.
So gilt \zB{}, dass $\gls{mu}_{00} = \gls{M}_{00}$ oder $\gls{mu}_{11} = \gls{M}_{11} - \bar{x}\gls{M}_{01} = \gls{M}_{11} - \bar{y}\gls{M}_{10}$ (\vgl{}~\cite{momente}).

\paragraph{Skalierungsinvariante Momente}
\label{skalierungsinvariante_zentrierte_momente}

Für $i+j \geq 2$ können desweiteren die \emph{skalierungsinvarianten Momente} $\gls{eta}_{ij}$ konstruiert werden, die invariant \bzgl{} Skalierung und Translation sind.
Dafür wird das entsprechende translationsinvariante Moment $\gls{mu}_{ij}$ durch die entsprechende Fläche $\gls{M}_{00}$ \bzw{} $\gls{mu}_{00}$ des Segments geteilt, \dhe{}~\cite{momente}
\begin{equation*}
  \gls{eta}_{ij} = \frac{\gls{mu}_{ij}}{\gls{mu}_{00}^{\left(1+\left(i+j\right)/2\right)}}.
\end{equation*}

\paragraph{(Rotationsinvariante) Hu-Momente}
\label{rotationsinvariante_momente}

\citeauthor{Hu} verwendet eine nichtlineare Kombination der skalierungsinvarianten Momente bis zum Grad $3$, um aus ihnen zusätzlich eine Rotationsinvarianz zu gewinnen~\cite{Hu, momente}.
Daraus ergeben sich sieben \emph{rotationsinvariante Momente} oder \emph{Hu-Momente}.
\begin{equation*}
\begin{split}
  \gls{hu}_1 & = \gls{eta}_{20} + \gls{eta}_{02}\\
  \gls{hu}_2 & = {\left(\gls{eta}_{20} - \gls{eta}_{02}\right)}^2 + {\left(2\gls{eta}_{11}\right)}^2\\
  \gls{hu}_3 & = {\left(\gls{eta}_{30} - 3\gls{eta}_{12}\right)}^2 + {\left(3\gls{eta}_{21} - \gls{eta}_{03}\right)}^2\\
  \gls{hu}_4 & = {\left(\gls{eta}_{30} + \gls{eta}_{12}\right)}^2 + {\left(\gls{eta}_{21} + \gls{eta}_{03}\right)}^2\\
  \gls{hu}_5 & = \left(\gls{eta}_{30} - 3\gls{eta}_{12}\right)\left(\gls{eta}_{30} + \gls{eta}_{12}\right) \left({\left(\gls{eta}_{30} + \gls{eta}_{12}\right)}^2 - 3{\left(\gls{eta}_{21} + \gls{eta}_{03}\right)}^2\right)\\
  & + \left(3\gls{eta}_{21} - \gls{eta}_{03}\right)\left(\gls{eta}_{21} + \gls{eta}_{03}\right) \left(3{\left(\gls{eta}_{30} + \gls{eta}_{12}\right)}^2 - {\left(\gls{eta}_{21} + \gls{eta}_{03}\right)}^2\right)\\
  \gls{hu}_6 & = \left(\gls{eta}_{20} - \gls{eta}_{02}\right)\left({\left(\gls{eta}_{30} + \gls{eta}_{12}\right)}^2 - {\left(\gls{eta}_{21} + \gls{eta}_{03}\right)}^2\right) + 4\gls{eta}_{11}\left(\gls{eta}_{30} + \gls{eta}_{12}\right)\left(\gls{eta}_{21} + \gls{eta}_{03}\right)\\
  \gls{hu}_7 & = \left(3\gls{eta}_{21} - \gls{eta}_{03}\right)\left(\gls{eta}_{30} + \gls{eta}_{12}\right) \left({\left(\gls{eta}_{30} + \gls{eta}_{12}\right)}^2 - 3{\left(\gls{eta}_{21} + \gls{eta}_{03}\right)}^2\right)\\
  & + \left(\gls{eta}_{03} - 3\gls{eta}_{12}\right)\left(\gls{eta}_{21} + \gls{eta}_{03}\right) \left(3{\left(\gls{eta}_{30} + \gls{eta}_{12}\right)}^2 - {\left(\gls{eta}_{21} + \gls{eta}_{03}\right)}^2\right)
\end{split}
\end{equation*}

\paragraph{Interpretation}
\label{interpretation_momente}

\todo{kovarianzmatrix \bzw{} inertia tensor}

\paragraph{Weitere Merkmale}
\label{weitere_merkmale}

\paragraph{Caching}
\label{Caching}

\cite{Siedhoff}

\section{Superpixel}
\label{superpixel}

Eine Superpixelrepräsentation eines Bildes
Die Segmentierungsmaske $\gls{S} \in {\left\{1, \ldots, S\right\}}^{H \times W}$ eines Bildes $\gls{B} \in {\left[0, 1\right]}^{H \times W \times C}$, wobei $S \in \gls{N}$ die Anzahl an Segmenten beschreibt.
Massefunktion $\gls{m} \colon \gls{V} \to \gls{N}$
$\gls{G} = \left(\gls{V}, \gls{E}, \gls{p}, \gls{m}\right)$

\paragraph{Globale Normierung}
\label{globale_normierung}

\paragraph{Lokale Normierung}
\label{globale_normierung}

\subsection{Verfahren}
\label{superpixel_verfahren}

\paragraph{SLIC}
\label{slic}

\cite{slic}

\emph{Simple Linear Iterative Clustering} (SLIC)

\paragraph{Quickshift}
\label{quickshift}

\cite{quickshift}

\paragraph{Weitere Verfahren}
\label{weitere_superpixel_verfahren}

\cite{felzenszwalb}

\subsection{Adjazenzmatrixbestimmung}
\label{adjazenzmatrixbestimmung}

\subsection{Merkmalsextraktion}
\label{merkmalsextraktion}

Aufgrund der geringeren Menge an Knoten über die Darstellung eines Bildes über die Superpixelrepräsentation als Graph benötigen
\cite{Siedhoff}
\cite{momente}

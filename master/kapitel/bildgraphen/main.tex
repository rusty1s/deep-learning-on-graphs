\chapter{Graphrepräsentationen von Bildern}
\label{graphrepraesentationen_von_bildern}

Als eine \emph{Graphrepräsentation eines Bildes} $\gls{B} \in {\left[0, 1\right]}^{H \times W \times C}$ wird eine Darstellung von \gls{B} als ein gewichteter, ungerichteter sowie schleifenloser Graph \gls{G} verstanden, deren Knoten Informationen zu ausgewählten Bereichen von \gls{B} über eine Merkmalsmatrix $\ma{F} \in \gls{R}^{N \times M}$ speichern und deren Kanten eine Aussage über die örtlichen Nachbarschaften eines jeden Bildbereichs inne wohnt.
Formal lässt sich eine Graphrepräsentation eines Bildes damit als ein \emph{Graph im zweidimensionalen euklidischen Raum} $\gls{G} = \left(\gls{V}, \gls{E}, \gls{p}\right)$ verstehen, dem zusätzlich zu seinen Knoten- und Kantenmengen anstatt einer Gewichtsfunktion $\gls{w} \colon \gls{E} \to \gls{R}$ eine Positionsfunktion $\gls{p} \colon \gls{V} \to \gls{R}^2$ auf seinen Knoten in den zweidimensionalen euklidischen Raum $\gls{R}^2$ zugeordnet ist.
Das Gewicht $\gls{w} \colon \gls{E} \to \left[0, 1\right]$ einer Kante ergibt sich dann implizit als \enquote{Abstandsfunktion} mit Hilfe von \gls{p} und der euklidischen Norm $\left\|\gls{p}\left(\gls{v}_i\right) - \gls{p}\left(\gls{v}_j\right) \right\|_2 \coloneqq \sqrt{{\left({\gls{p}\left(\gls{v}_i\right)}_1 - {\gls{p}\left(\gls{v}_j\right)}_1\right)}^2 + {\left({\gls{p}\left(\gls{v}_i\right)}_2 - {\gls{p}\left(\gls{v}_j\right)}_2\right)}^2}$ als
\begin{equation}
  \gls{w}\left(\gls{v}_i, \gls{v}_j\right) \coloneqq \begin{cases}
    \exp\left(-\frac{\left\|\gls{p}\left(\gls{v}_i\right) - \gls{p}\left(\gls{v}_j\right)\right\|_2^2}{2\gls{sigma}^2}\right), & \text{wenn }\left(\gls{v}_i, \gls{v}_j\right) \in \gls{E}\\
    0, & \text{sonst},
  \end{cases}
  \label{gauss}
\end{equation}
wobei die \emph{Gaußfunktion} $\exp\left(-x^2 / 2 \gls{sigma}^2\right)$ den Abstand zweier Knoten mit Hilfe eines festen Parameters $\gls{sigma} \in \gls{R}$ zueinander invertiert, sodass Knoten die weiter von einander entfernt liegen ein geringeres Gewicht besitzen~\cite{Shuman}.
Das korrespondiert mit der
Abbildung 5.1 veranschaulicht den Invertierungsprozess anhand unterschiedlich gewählter $\theta$.
Aufgrund der Symmetrie von ${\left\|\cdot\right\|}_2$ folgt damit sofort die Ungerichtheit des Graphen \gls{G}, \dhe{} $\gls{w}\left(\gls{v}_i, \gls{v}_j\right) = \gls{w}\left(\gls{v}_j, \gls{v}_i\right)$.

Damit lässt sich die Korrespondieren adjazenzmatrik Adist aus g() über
Zusätzlich lässt sich die Richtung der Kanten eines Graphen

Es ist insbesondere anzumerken, dass die winkelfubktion im gegensatz zur abstandsfunktion w nicht symmetrisch ist, d.h. ...
Weiterhin bildet die Winkelfunktion



Sooo
Graphen im zweidimensionalen raum
Definieren p, arad adist
Gaus

Gitter mit merkmalsfunktion f
Dann lokale oder globale normierung

Dann segmentierung


$\gls{G} = \left(\gls{V}, \gls{E}, \gls{p}\right)$

lokale oder globale Normierung

Ein \emph{ebener Graph} ist eine konkrete Darstellung eines Graphen auf der zweidimensionalen Ebene $\gls{R}^2$.
Jedem Knoten $v$ ist eine Positionsfunktion $\gls{p} \colon \gls{V} \to \gls{R}^2$ zugeordnet, die die Position eines Knotens auf der Ebene eindeutig definiert.

\gls{Adist} und \gls{Arad} machen Graphen translationsinvariant und abhängig von Sigma auch skalierungsinvariant


\section{Gitter}
\label{gitter}

Die einfachste Form einer Graphrepräsentation \gls{G} eines Bildes $\gls{B} \in \gls{R}^{H \times W \times C}$ ist die Repräsentation des Bildes über einen regulären Gittergraphen, denn dafür müssen keine Berechnungen am Bild vorgenommen werden.
Ein \emph{regulärer Gittergraph im zweidimensionalen euklidischen Raum} ist ein Graph $\gls{G} = \left(\gls{V}, \gls{E}, \gls{p}\right)$, der aus einem regulären Gitter gewonnen wurde und demnach genau $N \coloneqq H \times W$ Knoten enthält, \dhe{} einen Knoten für jedes Pixel in \gls{B}~\cite{Defferrard}.
Die Positionsfunktion der Knoten $\gls{p} \colon \gls{V} \to \gls{R}^2$ entspricht damit genau der Koordinate des korrespondierenden Pixels im Urpsprungsbild.
Sei dafür $\gls{v} \in \gls{V}$ der Knoten zu dem Bildpunkt an Position $\left(x,y\right)$.
Folglich gilt für die Position des Knotens $\gls{p}\left(\gls{v}\right) \coloneqq {\left[x,y\right]}^{\top}$.
Die örtlich um den Knoten $\gls{v} \in \gls{V}$ liegenden Knoten gelten dann auch im Gittergraph als benachbart und werden über eine Kante in \gls{E} verbunden.
Dabei unterscheidet man zwischen zwei frei wählbaren \emph{Konnektivitäten} des Graphen~\cite{Defferrard}.
Eine Konnektivität von $4$ bedeutet, dass ein Knoten (ohne Berücksichtigung der Randknoten) genau vier Nachbarn besitzt, die horizontal und vertikal zu ihm stehen.
Bei einer Konnektivität von $8$ gelten zusätzlich die vier diagonal zu ihm stehenden Knoten als Nachbarn und die Nachbarschaft wird folglich als ein $3 \times 3$ Fenster um den Knoten \gls{v} verstanden.

Analog zu den Daten der Farbkanäle an den einzelnen Pixeln des Bildes besitzt auch der Graph über einer Merkmalsfunktion $f \colon \gls{V} \to \gls{R}^C$ \bzw{} einer Merkmalsmatrix $\gls{F} \in \gls{R}^{N \times C}$ diese Information mit $f\left(\gls{v}\right) \coloneqq \gls{B}_{{p\left(\gls{v}\right)}_2,{p\left(\gls{v}\right)}_1}$ an seinen Knoten.

\paragraph{Effiziente Adjazenzbestimmung}
\label{adjazenzbestimmung_gitter}

Entgegen der Pixelanordnung in einem Bild ist die Anordnung der Knoten in einem Gittergraphen völlig irrelevant und kann willkürlich gewählt werden.
Ein Aufbau der Kantenmenge \gls{E} \bzw{} der korrespondierenden, ungewichteten Adjazenzmatrix \gls{A} kann jedoch besonders effizient gestaltet werden, wenn die Knoten zeilenweise entsprechend ihrer Bildkoordinate angeordnet werden.
Dafür wird das Bild \gls{B} zunächst an dessen Rändern um eine zusätzliche Spalte \bzw{} Zeile erweitert, \dhe{} $\gls{B} \in \gls{R}^{\left(H + 2\right) \times \left(W + 2\right) \times C}$.
Daraus ergeben sich $N \coloneqq \left(H + 2\right)\left(W + 2\right)$ viele Knoten eines Graphen, die die jeweiligen Gitterpunkte repräsentieren.
Ein Knoten $\gls{v}_i \in \gls{V}$, $W+3 < i < N - W - 3$, ist dann adjazent zu den Knoten $\gls{v}_j \in \gls{V}$ mit
\begin{equation*}
  j \in \left\{i-W-2, i-1, i+1, i+W+2\right\}
\end{equation*}
bei einer Konnektivität von $4$ \bzw{} mit
\begin{equation*}
  j \in \left\{i-W-3, i-W-2, i-W-1, i-1, i+1, i+W+1, i+W+2, i+W+3\right\}
\end{equation*}
bei einer Konnektivität von $8$.
Anschließend müssen die ungültigen Knoten, \dhe{} die an den Rändern des Gitters hinzugefügten Knoten, aus \gls{V} gelöscht werden.
Dazu zählen die ersten und letzten $W + 3$ Knoten aus \gls{V} sowie die vertikal am Rand des Gitters liegenden Knoten, die über eine Schrittweite von $W+2$ über der Knotenmenge eliminiert werden können.

\section{Superpixel}
\label{superpixel}

$\gls{G} = \left(\gls{V}, \gls{E}, \gls{p}, \gls{m}\right)$
Dann lokale oder globale normierung

Dann segmentierung

\subsection{Verfahren}
\label{superpixel_verfahren}

\paragraph{SLIC}
\label{slic}

\cite{slic}

\emph{Simple Linear Iterative Clustering} (SLIC)

\paragraph{Quickshift}
\label{quickshift}

\cite{quickshift}

\paragraph{Weitere Verfahren}
\label{weitere_superpixel_verfahren}

\cite{felzenszwalb}

\subsection{Adjazenzmatrixbestimmung}
\label{adjazenzmatrixbestimmung}

\subsection{Merkmalsextraktion}
\label{merkmalsextraktion}


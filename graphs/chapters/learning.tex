\section{Lernen von Graphen}

\subsection{Knotenauswahl}

\begin{itemize}
  \item Auswahl an Knoten, für die ein Receptive Field erstellt werden soll
  \item Sortierung soll dem Verfahren von Bildern nahekommen, d.h.\ Knoten mit ähnlichen strukturellen Merkmalen sollen auch in der Vektorrepräsentation nah beieinanderliegen
  \item Graph-Beschreibung $l$ – \underline{Metriken}:
  \begin{itemize}
    \item Grad der Knoten, d.h.\ Anzahl adjazenter Knoten (\underline{gewichtet:} Auswärtsgrad – Einwärtsgrad)
    \item \emph{Betweeness centrality}: $g(v) = \sum_{s \neq v \neq t} \frac{\sigma_{st}(v)}{\sigma{st}}$, wobei $\sigma_{st}$ die Anzahl an kürzesten Pfaden von $s$ nach $t$ ist und $\sigma_{st}$ die Anzahl dieser Pfade, die durch $v$ gehen
    \item \emph{Weisfeiler-Lehman Algorithmus}
    \item \emph{Page-Rank}
    \item \emph{Eigenvektor-Zentralität}
  \end{itemize}
  \item \underline{Gegeben:} Graph-Beschreibung $l$, Abstand $s$, Anzahl $w$ an Reciptive Fields
\end{itemize}

\begin{enumerate}
  \item sortiere die Knoten auf Basis von $l$
  \item iteriere über die sortierte Knotenmenge mit Abständen $s$, bis $w$ Knoten ausgewählt wurden
\end{enumerate}

\subsection{Nachbarschaftssuche}

\begin{itemize}
  \item \underline{Gegeben:} Knoten $v$, Größe $k$ des Receptive Fields
\end{itemize}

\begin{enumerate}
  \item setze initiale Knotenmenge $N$ auf $v$
  \item wiederhole bis $|N| > k$:
    \begin{enumerate}
      \item berechne für alle Knoten $i$ in $N$ die Nachbarschaften $N_1(i)$ und füge sie zu $N$ hinzu
    \end{enumerate}
\end{enumerate}

\begin{itemize}
  \item \underline{Bemerkung:} im Allgemein gilt $|N| \neq k$
\end{itemize}

\subsection{Normalisierung}

\begin{itemize}
  \item Aus einem Nachbarschaftsgraphen soll ein Receptive Field konstruiert werden
  \item Knoten werden anhand eines Graph-Labelings $l$ sortiert
  \begin{itemize}
    \item ein Receptive Field für die Knoten (Größe $k$) und ein Receptive Field für die Kanten (Größe $k \times k$)
    \item jedes Knoten- oder Kantenattribut wird in einem Receptive Field abgespeichert (z.B.\ Farbe)
  \end{itemize}
  \item \underline{Gegeben:} Menge von Graphen $\mathcal{G}$ mit $k$ Knoten, Distanzmetriken für $k \times k$ Matrizen $d_A$ und Graphen $d_G$ für $k$ Knoten
  \begin{itemize}
    \item $d_A$, z.B.\ \emph{Hamming-Abstand}: $d_A(x, y) = | \lbrace j \in \lbrace 1, \ldots, N \rbrace | x_j \neq y_j \rbrace |$
    \item \underline{Beispiel:} $12345$ und $13344 \rightarrow 2$
    \item $d_G$: z.B.\ \emph{Edit distance}
  \end{itemize}
  \item \underline{Optimierungsproblem über $l$:} $\min_l \sum_{G \in \mathcal{G}} \sum_{G' \in \mathcal{G}} {( d_A(A^l(G), A^l(G') - d_g(G, G')) )}$
  \item $\Rightarrow$ für beliebige Graphen $G$ und $G'$ soll die Ähnlichkeit dieser Graphen gleich der Ähnlichkeit der Graphen im Vektorraum sein (basierend auf den Adjazenzmatrizen der Graphen)
  \item $\Rightarrow$ Problem is NP-schwer
  \item \underline{Alternative:} wähle aus einer Menge von Labelings die beste zu einer gegebenen Menge von Graphen
  \begin{itemize}
    \item $\lbrace (G_1, G'_1), \ldots, (G_N, G'_N)$ eine zufällge Auswahl an Graphpaaren von $\mathcal{G}$
    \item wähle das Labeling $l$ so, dass $\sum_{i=1}^N \frac{d_A(A^l(G_i), A^l(G'_i))}{N}$ minimal
  \end{itemize}
  \item Labelings werden nur berechnet für Knoten gleicher Distanz zum Startknoten $v$
  \item Labelings sind im Allgemeinen nicht injektiv $\Rightarrow$ sortiere anhand lexikographischer maximaler Adjazenzmatrizen
\end{itemize}

\begin{figure}[h]
  \centering
  \includegraphics[width=.9\textwidth]{images/normalization}
\end{figure}

\section{Dünnbesetzte Matrizen}

Aus Speichereffizienz-Gründen lohnt es sich mit \emph{dünnbesetzte Matrizen} (engl. \emph{Sparse Matrices}) für Adjazenzmatrizen zu rechnen.
Eine dünnbesetzte Matrix ist eine Matrix, bei der so viele Einträge aus Nullen bestehen, dass man nach Möglichkeiten sucht, dies insbesondere hinsichtlich Algorithmen sowie Speicherung auszunutzen.
In der Regel gelten Matrizen als dünnbesetzt, wenn eine Matrix mit $n^2$ Einträgen nicht mehr als $n$ oder $n \cdot \log n$ Einträge ungleich Null besitzt.
Eine Matrix die nicht dünnbesetzt ist, heißt \emph{dicht} (engl. \emph{dense}).

Dünnbesetzte Matrizen erlauben einigen Operationen, optimierter bzw.\ schneller zu sein, da nicht alle Werte der Matrix betrachtet werden müssen.
Es gibt jedoch auch Operatione, die nicht auf dünnbesetzten Matritzen definiert sind, so dass sie vorher in eine normale Dense-Matrix überführt werden müssen.
